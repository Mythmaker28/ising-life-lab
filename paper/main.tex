\documentclass[12pt]{article}

% Standard packages for arXiv
\usepackage{amsmath,amssymb}
\usepackage{graphicx}
\usepackage[utf8]{inputenc}
\usepackage{authblk}
\usepackage[colorlinks=true,linkcolor=blue,citecolor=blue,urlcolor=blue]{hyperref}
\usepackage{geometry}
\geometry{margin=1in}

\title{Geometric Control of Biological Quantum Sensors: \\
       An Open-Source Pipeline Based on Ising-Life-Lab and a Multi-Modal Atlas}

\author[1]{T. Mythmaker}
\affil[1]{Independent Researcher}

\date{\today}

\begin{document}

\maketitle

\begin{abstract}
Robust control of biological quantum sensors operating under physiological conditions remains a key challenge in quantum biosensing. We present an open-source computational pipeline that bridges a multi-modal atlas of biological quantum systems with a phase-coupled oscillator control engine (Ising-Life-Lab). We compare dynamic control (linear ramping, P3) against geometric control (closed-loop trajectories with Berry phase accumulation, P4) across three representative biological quantum systems: nitrogen-vacancy (NV) centers at room temperature (298K), hyperpolarized $^{13}$C pyruvate for metabolic imaging (310K), and radical pair cryptochrome for avian magnetoreception (310K). Our results show that, in this toy Kuramoto-XY model, dynamic control (P3) achieves lower robustness costs (0.00 vs 0.01--0.10) despite the theoretical expectation of topological protection in geometric trajectories. However, P4 exhibits marginally better stability variance ($\sim$1.1$\times$). These findings highlight the gap between idealized geometric control theory and practical implementation in noisy biological regimes, and demonstrate the value of reproducible computational pipelines for quantum biosensing research. All code, data, and figures are available at \url{https://github.com/Mythmaker28/ising-life-lab}.
\end{abstract}

\section{Introduction}

Biological quantum sensors—including nitrogen-vacancy (NV) centers in diamond, hyperpolarized nuclear spins in metabolites, and radical pairs in cryptochrome proteins—offer unprecedented sensitivity for probing biological processes at the nanoscale \cite{Schirhagl2014,Ardenkjaer2020,Hore2016}. However, operating these sensors under physiological conditions (room temperature, aqueous environments, biological noise) presents major challenges: short coherence times ($T_2 \sim 0.8$--3.5 $\mu$s for our systems), thermal fluctuations, and environmental decoherence.

Control theory provides tools to optimize sensor performance under constraints. \textit{Dynamic control} strategies (e.g., linear ramping of coupling parameters) are straightforward but sensitive to noise. \textit{Geometric control}, leveraging Berry phase and topological properties of closed-loop trajectories in parameter space \cite{Berry1984,Sjoqvist2012}, promises inherent robustness due to the path's geometric properties being insensitive to small perturbations.

Despite theoretical promise, few studies have systematically compared these strategies across diverse biological quantum modalities. This gap motivated our work: we developed an open-source pipeline linking:
\begin{enumerate}
    \item A \textbf{multi-modal atlas} of biological quantum systems (180+ optical, spin, nuclear, and radical pair sensors),
    \item A \textbf{phenomenological control engine} (Ising-Life-Lab) based on coupled Kuramoto-XY oscillators,
    \item Automated comparison of P3 (dynamic) vs P4 (geometric) control with quantitative metrics.
\end{enumerate}

This paper presents the first implementation and benchmark, focusing on three representative systems. We find that the toy model exhibits \textit{counter-intuitive} results where dynamic control outperforms geometric control, highlighting the importance of model fidelity and the need for experimental validation.

\section{Background}

\subsection{Biological Quantum Sensors}

\textbf{NV Centers at Room Temperature.} 
Nitrogen-vacancy defects in diamond are optically addressable spin qubits with potential for biosensing via magnetic field detection \cite{Schirhagl2014}. At 298K, coherence times drop to $T_2 \sim 1.8$ $\mu$s due to phonon interactions and surface-induced decoherence.

\textbf{Hyperpolarized $^{13}$C Pyruvate.}
Dissolution dynamic nuclear polarization (DNP) enhances $^{13}$C nuclear spin polarization by $>10^4\times$, enabling real-time metabolic imaging \cite{Ardenkjaer2020}. Operating at $\sim$310K, these systems have $T_2 \sim 3.5$ $\mu$s in vivo.

\textbf{Radical Pair Cryptochrome (Cry4).}
Cryptochrome proteins in birds may mediate magnetoreception through spin-correlated radical pairs \cite{Hore2016}. These systems operate at 310K with extremely short $T_2 \sim 0.8$ $\mu$s due to rapid recombination dynamics.

\subsection{Dynamic vs Geometric Control}

\textbf{Dynamic Control (P3).}
In P3, control parameters (e.g., coupling strength $K$, noise amplitude, annealing rate) evolve via deterministic time-dependent functions:
\begin{equation}
K(t) = K_{\mathrm{start}} + \frac{K_{\mathrm{end}} - K_{\mathrm{start}}}{T} t, \quad t \in [0, T]
\end{equation}
This open-path strategy is computationally simple but offers no inherent protection against perturbations.

\textbf{Geometric Control (P4).}
In P4, parameters trace a \textit{closed loop} in parameter space (e.g., $(K_1, K_2)$):
\begin{equation}
\begin{pmatrix} K_1(t) \\ K_2(t) \end{pmatrix} = 
\begin{pmatrix} K_1^c \\ K_2^c \end{pmatrix} + 
\begin{pmatrix} r_1 \cos(2\pi t/T) \\ r_2 \sin(2\pi t/T) \end{pmatrix}
\end{equation}
The system accumulates a \textit{geometric phase} (Berry phase):
\begin{equation}
\gamma_{\mathrm{Berry}} = \oint_{\mathcal{C}} \mathbf{A} \cdot d\mathbf{r}
\end{equation}
where $\mathbf{A}$ is the Berry connection. Geometric phases are robust to small path deformations—a form of topological protection \cite{Berry1984}.

\section{Methods}

\subsection{Data Bridge: Atlas $\rightarrow$ Ising-Life-Lab}

We developed a data bridge (\texttt{isinglab.data\_bridge}) that maps physical properties from the biological quantum atlas to phenomenological parameters:
\begin{itemize}
    \item \textbf{Atlas inputs}: System ID, $T_1$, $T_2$, temperature, modality.
    \item \textbf{Phenomenological outputs}: Maximum coupling $K_{\max}$, noise amplitude, annealing schedule.
\end{itemize}

\textbf{Mapping heuristics:}
\begin{align}
K_{\max} &\propto \frac{T_1}{T_2} \quad \text{(coupling limited by coherence)} \\
\sigma_{\mathrm{noise}} &\propto \frac{1}{T_2} \quad \text{(shorter $T_2$ $\Rightarrow$ more noise)}
\end{align}

For this study, we used a mock atlas (\texttt{atlas\_mock.csv}) with three curated systems:

\begin{table}[h]
\centering
\begin{tabular}{lccc}
\hline
\textbf{System} & \textbf{$T_2$ ($\mu$s)} & \textbf{Temp (K)} & \textbf{Noise} \\
\hline
NV-298K & 1.8 & 298 & 0.150 \\
C13-Pyruvate & 3.5 & 310 & 0.180 \\
RP-Cry4 & 0.8 & 310 & 0.250 \\
\hline
\end{tabular}
\caption{Physical properties of the three biological quantum systems studied.}
\label{tab:systems}
\end{table}

\subsection{Control Strategies (P3 vs P4)}

Both strategies operate on a Kuramoto-XY oscillator lattice (64$\times$64 grid) with multi-kernel coupling:
\begin{equation}
\frac{d\theta_i}{dt} = \omega_i + \sum_{j \in \mathcal{N}_i} K_j \sin(\theta_j - \theta_i) + \xi_i(t) - \lambda |\nabla^2 \theta_i|
\end{equation}
where $\omega_i$ are natural frequencies, $K_j$ are distance-dependent couplings, $\xi_i(t)$ is Gaussian noise, and $\lambda$ is an annealing damping term.

\textbf{P3 (Dynamic):} Linear ramp of $K_1$ from 0.5 to 0.85 over duration $T = 1.0$.

\textbf{P4 (Geometric):} Elliptical loop in $(K_1, K_2)$ space with 20 points, duration $T = 1.0$. The geometric phase $\gamma_{\mathrm{Berry}}$ was computed via:
\begin{equation}
\gamma_{\mathrm{Berry}} \approx \pi r_1 r_2 \quad \text{(for ellipse)}
\end{equation}

\textbf{Metrics:}
\begin{itemize}
    \item \textbf{Robustness cost}: Mean $L^2$ distance between clean and noisy trajectories' final states (order parameter $r$, defect density). Lower is better.
    \item \textbf{Stability variance}: Variance of final $r$ across multiple noise realizations. Lower is more stable.
\end{itemize}

\subsection{Experimental Pipeline (Notebook + Figures + CSV)}

The entire workflow is implemented in an executable Jupyter notebook (\texttt{ATLAS\_GEOMETRIC\_CONTROL\_DEMO.ipynb}):
\begin{enumerate}
    \item Load 3 systems via \texttt{AtlasMapper(mode='mock')}.
    \item Define baseline P3 parameters (system-specific ramps).
    \item Run \texttt{compare\_geometric\_vs\_dynamic\_robustness()} for each system ($n = 3$ noise trials).
    \item Generate 3 publication-ready figures + 1 CSV results table.
\end{enumerate}

All code is version-controlled at \url{https://github.com/Mythmaker28/ising-life-lab} (branch \texttt{toolkit-core-r1}).

\section{Results}

\subsection{NV Center: P3 vs P4}

Figure~\ref{fig:nv_p3_vs_p4} shows robustness and stability metrics for the NV-298K system.

\begin{figure}[h]
\centering
\includegraphics[width=\textwidth]{figures/figure1_nv298k_p3_vs_p4.png}
\caption{Dynamic (P3) vs geometric (P4) control on the NV-298K system. Left: Robustness cost (lower is better). Right: Stability variance (lower is more stable). P3 achieves zero robustness cost, while P4 exhibits non-zero cost despite geometric phase accumulation ($\gamma = 0.173$ rad).}
\label{fig:nv_p3_vs_p4}
\end{figure}

\textbf{Key findings:}
\begin{itemize}
    \item P3 robustness cost: $0.0000 \pm 0.0000$
    \item P4 robustness cost: $0.0984 \pm 0.0921$
    \item \textbf{Winner: P3} (100\% improvement, i.e., P4 worse)
    \item Stability variance: P4 marginally better ($1.14\times$ reduction)
\end{itemize}

\subsection{Cross-Modal Comparison (NV vs $^{13}$C vs Radical Pair)}

Figure~\ref{fig:multisystem} compares P3 and P4 performance across all three systems.

\begin{figure}[h]
\centering
\includegraphics[width=\textwidth]{figures/figure2_multi_system_comparison.png}
\caption{Comparison across NV, $^{13}$C pyruvate, and radical pair cryptochrome systems. Left: Robustness cost by system and strategy. Right: P4 improvement over P3 (negative values indicate P3 wins). All three systems favor P3 in this toy model.}
\label{fig:multisystem}
\end{figure}

\textbf{Observations:}
\begin{itemize}
    \item P3 outperforms P4 on \textbf{all} systems.
    \item $^{13}$C pyruvate (highest $T_2 = 3.5$ $\mu$s) shows smallest P4 cost ($0.0001$), but still worse than P3.
    \item RP-Cry4 (lowest $T_2 = 0.8$ $\mu$s) has highest P4 cost ($0.0689 \pm 0.0974$).
\end{itemize}

Figure~\ref{fig:properties} summarizes the physical properties and geometric phases of the three systems.

\begin{figure}[h]
\centering
\includegraphics[width=\textwidth]{figures/figure3_system_properties.png}
\caption{Physical properties of the three biological quantum systems. Top left: Coherence time $T_2$. Top right: Operating temperature. Bottom left: Intrinsic noise level. Bottom right: Geometric phase (Berry phase) accumulated by P4 trajectories. Systems with higher $T_2$ have larger geometric phases due to tighter control loops.}
\label{fig:properties}
\end{figure}

\subsection{Limitations of the Toy Model}

The counter-intuitive result (P3 $>$ P4) stems from:
\begin{enumerate}
    \item \textbf{Model fidelity}: The Kuramoto-XY lattice is a \textit{phenomenological} abstraction. Real quantum sensors have distinct Hamiltonians (spin-1, nuclear spin $I = 1/2$, radical pair spin dynamics) not captured here.
    \item \textbf{Noise model}: We used uncorrelated Gaussian noise. Biological noise (e.g., 1/f, shot noise) may behave differently.
    \item \textbf{Parameter tuning}: P3 parameters were hand-tuned; P4 loop shapes were generic ellipses. Optimized P4 trajectories (via Bayesian optimization) may perform better.
    \item \textbf{Absence of dissipation channels}: Real systems have $T_1$-limited energy relaxation, not modeled explicitly here.
\end{enumerate}

\section{Discussion}

\subsection{Interpreting the Results}

The finding that dynamic control (P3) outperforms geometric control (P4) in this toy model is \textit{surprising} but \textit{instructive}. It suggests:
\begin{itemize}
    \item Geometric phases provide robustness to \textit{slow, adiabatic} perturbations but may not protect against \textit{fast, stochastic} noise.
    \item The benefit of geometric control may only manifest in higher-fidelity models or experimental systems where adiabaticity conditions are met.
    \item Phenomenological models (like ours) can guide intuition but require validation against quantum master equations or experiments.
\end{itemize}

\subsection{Implications for Quantum Biosensing}

For practitioners designing control protocols for biological quantum sensors:
\begin{enumerate}
    \item \textbf{Start simple}: Linear ramps (P3) may suffice if noise is non-adversarial.
    \item \textbf{Test geometric control experimentally}: Our toy model's failure does not invalidate geometric control—it highlights the need for real-world tests.
    \item \textbf{Use multi-modal atlases}: Systematic comparison across NV, $^{13}$C, and radical pairs reveals modality-specific challenges (e.g., RP-Cry4's ultra-short $T_2$).
\end{enumerate}

\subsection{Related Work}

Geometric control has been experimentally validated in superconducting qubits \cite{Abdumalikov2013} and trapped ions \cite{Duan2001} under cryogenic, isolated conditions. Our study extends this to \textit{warm, noisy, biological} contexts—a regime where theory and practice diverge.

\section{Future Work}

\begin{enumerate}
    \item \textbf{Scale to full atlas}: The mock atlas includes only 3 systems; the full atlas has 180+ systems. Batch processing will reveal statistical trends.
    \item \textbf{Upgrade model fidelity}: Replace Kuramoto-XY with spin-1 Hamiltonians (NV), Bloch equations ($^{13}$C), or radical pair Hamiltonian (Cry4).
    \item \textbf{Bayesian optimization of P4}: Current P4 loops are generic ellipses. Optimizing loop shape, duration, and phase may improve performance.
    \item \textbf{Experimental validation}: Collaborate with experimental groups to test P3 vs P4 on real NV sensors or hyperpolarized MRI.
    \item \textbf{Noise spectroscopy}: Characterize biological noise (1/f, shot, thermal) and update the noise model accordingly.
\end{enumerate}

\section{Conclusion}

We presented an open-source computational pipeline that bridges a multi-modal biological quantum atlas with a phenomenological control engine. Comparing dynamic (P3) and geometric (P4) control across three biological quantum systems (NV centers, $^{13}$C pyruvate, cryptochrome radical pairs), we found that P3 consistently outperformed P4 in this toy Kuramoto-XY model—contrary to theoretical expectations of topological protection.

These results underscore the importance of:
\begin{itemize}
    \item \textbf{Model validation}: Phenomenological models guide intuition but must be validated against quantum master equations and experiments.
    \item \textbf{Reproducibility}: Our pipeline (code, data, figures) is fully open-source, enabling independent verification and extension.
    \item \textbf{Honest reporting}: Negative or counter-intuitive results are valuable—they reveal gaps between theory and practice.
\end{itemize}

We invite the community to extend this work, improve model fidelity, and test these strategies experimentally. All code and data are available at \url{https://github.com/Mythmaker28/ising-life-lab}.

\section*{Acknowledgments}

This work was conducted independently with no external funding. We thank the open-source scientific Python community (NumPy, SciPy, Matplotlib, Pandas, Numba) for enabling this research.

\section*{Data and Code Availability}

All code, notebooks, figures, and data are publicly available at:
\begin{itemize}
    \item Repository: \url{https://github.com/Mythmaker28/ising-life-lab}
    \item Branch: \texttt{toolkit-core-r1}
    \item Notebook: \texttt{notebooks/ATLAS\_GEOMETRIC\_CONTROL\_DEMO.ipynb}
    \item Figures: \texttt{figures/atlas\_geometric\_demo/}
\end{itemize}

\begin{thebibliography}{99}

\bibitem{Schirhagl2014}
R. Schirhagl, K. Chang, M. Loretz, and C. L. Degen,
``Nitrogen-vacancy centers in diamond: nanoscale sensors for physics and biology,''
\textit{Annu. Rev. Phys. Chem.} \textbf{65}, 83--105 (2014).

\bibitem{Ardenkjaer2020}
A. Ardenkjær-Larsen et al.,
``Hyperpolarized water for metabolic MRI,''
\textit{Proc. Natl. Acad. Sci. USA} \textbf{117}(22), 11902--11906 (2020).

\bibitem{Hore2016}
P. J. Hore and H. Mouritsen,
``The radical-pair mechanism of magnetoreception,''
\textit{Annu. Rev. Biophys.} \textbf{45}, 299--344 (2016).

\bibitem{Berry1984}
M. V. Berry,
``Quantal phase factors accompanying adiabatic changes,''
\textit{Proc. R. Soc. Lond. A} \textbf{392}, 45--57 (1984).

\bibitem{Sjoqvist2012}
E. Sjöqvist et al.,
``Geometric phases for non-degenerate and degenerate mixed states,''
\textit{J. Phys. A: Math. Theor.} \textbf{45}, 145301 (2012).

\bibitem{Abdumalikov2013}
A. A. Abdumalikov Jr. et al.,
``Experimental realization of non-Abelian non-adiabatic geometric gates,''
\textit{Nature} \textbf{496}, 482--485 (2013).

\bibitem{Duan2001}
L.-M. Duan, J. I. Cirac, and P. Zoller,
``Geometric manipulation of trapped ions for quantum computation,''
\textit{Science} \textbf{292}, 1695--1697 (2001).

\end{thebibliography}

\end{document}

